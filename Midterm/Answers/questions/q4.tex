به جای پیدا کردن k امین عنصر و پس از آن مقایسه آن با x سعی می کنیم که با پیمایش هرم k عنصر کوچکتر از عدد x را بیابیم.
اگر موفق به این کار بشویم نتیجه میگیریم که عنصر k امین کوچکترین عنصر از x کوچکتر است. اگر نتوانیم در پیمایش خود k عنصر را بیابیم و همه عناصر کوچکتر از x را ویزیت کرده باشیم که تعدادشان کمتر از k است، در این صورت k امین کوچکترین عنصر از x بزرگتر است.

\begin{flushleft}

$def \hspace{5pt} function (node, x, k):$\\
    $\qquad if \hspace{3pt} node.value >= x:$\\
    $\qquad \qquad return \hspace{5pt} 0$\\
    $\qquad counter = 1$\\
	$\qquad if \hspace{3pt} counter == k:$\\
	$\qquad \qquad return \hspace{5pt} k$\\
	$\qquad if \hspace{3pt} node. left:$\\
	$\qquad \qquad counter \hspace{3pt} += \hspace{3pt} function (node.left,  \hspace{3pt} x,  \hspace{3pt} k-counter)$\\
	$\qquad if \hspace{3pt} counter == k:$\\
	$\qquad \qquad return \hspace{5pt} k$\\
	$\qquad if \hspace{3pt} node. right:$\\
	$\qquad \qquad counter \hspace{3pt} += \hspace{3pt} function(node.right,  \hspace{3pt} x,  \hspace{3pt} k-counter)$\\
	$\qquad return \hspace{5pt} counter$\\

 \end{flushleft}

 حال اگر خروجی این تابع نهایتا k باشد پس x از k امین عنصر کوچک هرم بزرگتر است و اما در غیر این صورت این مورد برقرار نیست.
 برای هر یک از k راس کوچکتر از x حداکثر دو فرزند داریم که تابع بازگشتی برای آن ها اجرا می شود اما جز رئوس کوچکتر از x نیستند. پس در نهایت حداکثر برای پیدا کردن k راس 3k بار تابع صدا زده می شود که هرکدام از
 $O(1)$
 می باشد پس در نهایت پیچیدگی زمانی نهایی ما از
  $O(k)$
  خواهد بود.
