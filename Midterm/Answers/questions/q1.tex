\begin{RTL}
\begin{itemize}
\item الف) فرض کنید $ n = 2^{m}$\\
\begin{LTR}
$T(2^{m}) = 8T(2^{\frac{m}{2}}) + m^{3}$ \\
\\
$S(m) = T(2^{m})$ $\Rightarrow$  $S(m) = 8S(\frac{m}{2}) + m^{3}$\\


\end{LTR}
مطابق قضیه‌ی اصلی چون $\log_2 8 = 3$ :\\
\begin{LTR}
$S(m) = \theta(m^{3} \log_2 m)$ $\Rightarrow$ $m = \log_2 n$
$\Rightarrow$  $T(n) = \theta((\log_2 n)^{3} \log_2 \log_2 n )$
\end{LTR}


\item ب) حدس می‌زنیم $T(n)= \mathcal{O}(n)$. از استقرا استفاده می‌کنیم. می‌دانیم برای همه‌ی 
$c\geq1$ پایه‌ی $T(n)= 1 \leq cn$ برای همه‌ی $n\leq 5 $ برقرار است. داریم:   \\
\begin{LTR}
$T(n) = T(\frac{n}{2}) + T(\frac{n}{4}) + n \leq \frac{cn}{2}+\frac{cn}{4} + n = (\frac{3c}{4}+1)n$
\end{LTR}

اگر$c = 4$  را انتخاب کنیم آنگاه $T(n)\leq 4n = cn$ می‌شود. پس برای همه‌ی $c\geq 4$ و $n\geq 1$ $\Leftarrow$ $T(n)\leq cn$ $\Leftarrow$  $T(n)= \mathcal{O}(n)$\\

\item ج)
\begin{LTR}
$nT(n) = \sum_{i = 1}^{n-1} T(i) + 3n^{2}$\\


$(n+1) T(n+1) = \sum_{i = 1}^{n} T(i) + 3(n+1)^{2}$

\end{LTR}
دو عبارت را از هم کم می‌کنیم :
\begin{LTR}
$(n+1)T(n+1) - nT(n) = T(n) + 6n + 3$ 

$\Rightarrow$ $T(n+1) = \frac{nT(n) + T(n)}{n+1} + \frac{6n+3}{n+1} = T(n) -\frac{3}{n+1} + 6 $ 

$\Rightarrow$ $T(n) = T(n-1) - \frac{3}{n} + 6$ $\Rightarrow$ $T(n) = T(n-2) + \frac{3}{n-1} + 6 + \frac{3}{n} + 6$

$\Rightarrow$ $T(n) = 6(n-2) + 3(\frac{1}{n} + \frac{1}{n-1} +...+ \frac{1}{2})$

$\Rightarrow$ $T(n) = \mathcal{O}(\max{(n,\log_2 n)})$

$\Rightarrow$ $T(n) = \mathcal{O}(n)$
\end{LTR}
\end{itemize}

\end{RTL}