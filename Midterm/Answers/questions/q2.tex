ابتدا توپ‌ها را مرتبا به دو بخش تقسیم می‌کنیم تا به آرایه‌های دوتایی برسیم. سپس به صورت بازگشتی آن‌ها را با هم مقایسه می‌کنیم. اگر بخش‌های تقسیم شده را کیسه‌های حاوی توپ در نظر بگیریم می‌دانیم در صورت وجود داشتن عضو اکثریت کیسه‌ای که حداقل  $\frac{n}{2}$ عضو داشته باشد شامل مجموعه‌ی جواب ما خواهد بود. چون در هنگام بازگشت فقط یک عضو از هر دو کیسه را با هم مقایسه کردیم دیگر نیاز به مقایسه‌ی تک به تک عضوها با هم نخواهد بود.
رابطه‌ی بازگشتی سوال به صورت زیر است که $\mathcal{O}(n)$ برای مرحله‌ی ادغام است:
\begin{LTR}
$T(n) = 2T(\frac{n}{2}) + O(n)$ 
\end{LTR}

مطابق با قضیه‌ی اصلی پیچیدگی الگوریتم از $\mathcal{O}(n\log n)$
می‌شود.
