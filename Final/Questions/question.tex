% !TeX program = xelatex
\input{template/packages}
\usepackage{template/template}
\usepackage{tabularx}
\usepackage{graphicx}
\graphicspath{ {./images/} }

\begin{document}
	\def\ci{\perp\!\!\!\perp}

	

	% header{<Assignment-Number>}{<Assignment-Title>}{<Deadline-Date>}{<Gathered-by>}{<Supervised-by>}
	\header
		{امتحان پایانترم}{}{}{}{}
	\input{template/info}

	% partheader{<Part-Name>}{<Part-Total-Score>}
	\partheader{سوالات }{100}

		نام و نام خانوادگی:\\
		شماره دانشجویی:\\
		\textbf{به موارد زیر توجه کنید.}
		
		\begin{itemize}
			\item مدت امتحان ۱۵۰ دقیقه است.
			\item امیدواریم که تا این جای کلاس با تفکر الگوریتمی آشنا شده باشید و با همین تفکر به سوالات پاسخ دهید.
		\end{itemize}
		
	\begin{enumerate}
		\setlength{\itemsep}{30pt}
		% question{<Question-TexFile-Path>}{<Question-Score>}
		\question{questions/q1}{10}
		\question{questions/q2}{10}
		\question{questions/q3}{15}
		\question{questions/q4}{10}
		\question{questions/q5}{15}
		\question{questions/q6}{10}
		\question{questions/q7}{10}
		\pagebreak
		\question{questions/q8}{20}

		\end{enumerate}
	\begin{flushleft}
			موفق باشید. 
	\end{flushleft}
\end{document}
