تعریف: به خانواده‌ای از \lr{Hash Function}ها \lr{$\epsilon$-universal} می‌گوییم که به ازای هر دو کلید $k_1$ و $k_2$ که از خانواده کلیدها انتخاب کنیم،
احتمال برخورد این دو کلید حداکثر برابر $\epsilon$ باشد.
\\
نشان دهید که در هر خانواده از \lr{$\epsilon$-universal hash functions} از مجموعه متناهی U (خانواده کلیدها) به مجموعه متناهی Q (که $|Q|=m$ و $m$ همان اندازه جدول درهم‌سازی می‌باشد) عبارت زیر همواره برقرار است:
\begin{gather*} \epsilon \leq 1/|Q| - 1/|U| \end{gather*}